% ! TeX root = ./document.tex
\documentclass[8pt, aspectratio=169, handout]{beamer}
\input{partials/package.tex}
% ! TeX root = ../document.tex
\usetheme{material}
\useLightTheme
\usePrimaryBlueGrey
\useAccentOrange

% ! TeX root = ../document.tex

\title{
  \textbf{Machine Learning for Aggregate Computing}
}
\subtitle{
  a Research Roadmap
}
\author[G.Aguzzi]{
  \textbf{Gianluca Aguzzi}\inst{1}, Roberto Casadei \inst{1}, Mirko Viroli \inst{1}
}
\institute{
  \inst{1}
  \texttt{Alma Mater Studiorum} -- Università di Bologna, Cesena, Italy
}
\talk{talk @ DISCOLI 2022}

\bibliography{biblio.bib}
\begin{document}
\begin{frame}[plain]
  \begin{backgroundblock} 
    \includegraphics[width=\paperwidth]{img/network.jpg} 
  \end{backgroundblock} 
\titlepage
\end{frame}
\addtocounter{framenumber}{-1}
\section{Background}
\begin{frame}{Context}
  \begin{card}[Engineering distributed Collectives Intelligence]
    \begin{itemize}
      \item \highlight{Why?} \faArrowRight \, distributed artificial system/socio-technical/cyber-physical deal with ``collectives''
      \begin{multicols}{2}
      \begin{itemize}
        \item Swarm robotics
        \item Crowd Engineering
        \item Smart cities/buildings/house
      \end{itemize}
      \begin{itemize}
        \item Wireless Sensor Networks
        \item Industry 4.0
        \item Traffic management
      \end{itemize}
      \end{multicols}
      \item Environment interactions \faPlus \, Large network of cyber-physical/artificial entities
    \end{itemize}
  \end{card}
  \begin{alarm}[ \faExclamationTriangle \, \textbf{How}?]
    \highlight{Engineering perspective} \faArrowRight \, Models, tools, simulations, abstractions, programming languages, \dots
  \end{alarm}
\end{frame}

\begin{frame}{A programming approach: Aggregate Computing~\cite{beal2015aggregate-programming}}
    \begin{columns}
      \begin{column}[c]{0.8\textwidth}
        \begin{card}[What?]
          \begin{itemize}
            \item[\success{\faInfo}] a \emph{macro-programming} paradigm for Collective Adaptive Systems (CASs)
            \begin{itemize}
              \item \highlight{macro-programming} \faArrowRight \, devising the macroscopic behaviour of systems of computational entities
              \item \highlight{Motto}: \emph{program the aggregate, not the individual devices}
              \item A way to \emph{program} Collective Intelligence \& Self-organisation 
            \end{itemize}
            \item[\success{\faInfo}] Collective Behaviour expressed through \emph{Computational fields} (short---\emph{fields})
            \begin{itemize}
              \item Distributed spacetime data structure \faArrowRight \, $F: (Devices, Time) \rightarrow (Values)$
            \end{itemize}
            \item[\success{\faInfo}] \highlight{Fields calculus}~\cite{audrito2019tocl} \faArrowRight \, core language to express fields manipulation
            \begin{itemize}
              \item Aggregate computing \faArrowRight \, Field calculus = \, Functional Programming \faArrowRight \, Lambda calculus
              \item function application (computation + branching)
              \item stateful evolution (\highlight{rep})
              \item local interaction (\highlight{nbr})
            \end{itemize}
          \end{itemize}
        \end{card}
      \end{column}
      \begin{column}[c]{0.2\textwidth}
        \cardImg{img/ac-vertical.png}{\textwidth}
      \end{column}
    \end{columns}
\end{frame}
\begin{frame}[allowframebreaks, fragile]{What can we express with Aggregate Computing?}
  \centering
  \cardImg{img/channel.png}{0.6\textwidth}
  \begin{card}[Aggregate Program specification]
    \begin{minted}{scala}
def channel(source: Boolean, target: Boolean, width: Double): Boolean = {
  distanceTo(source) + distanceTo(target) <= distanceBetween(source, target) + width
}
    \end{minted}
  \end{card}
  \cardImg{img/composition.png}{0.45\textwidth}
\end{frame}
\begin{frame}{Aggregate Computing: how?}
  \begin{card}[Requirements]
    \begin{itemize}
      \item \highlight{Structure}: network, based on some neighbouring relationship
      \item \highlight{Dynamics}: repeated, async execution of an ``aggregate'' protocol    
    \end{itemize}
  \end{card}
  \begin{card}[Execution Model]
    \begin{itemize}
      \item[\success{\faInfo}] \highlight{Asynchronous execution rounds}:
      \begin{itemize}
        \item [\highlight{1-}] \textbf{perceive} local context (messages from neighbours, sensed environment state) 
        \item [\highlight{2-}] \textbf{compute} \emph{aggregate program} against local context
        \item [\highlight{3-}] \textbf{act} on local context (send messages to neighbours, act on environment)
      \end{itemize}
    \end{itemize}
  \end{card}
\end{frame}
\begin{frame}{Aggregate Computing: Layered Approach}
  \centering
  \cardImg{img/aggregate-computing-stack.png}{0.5\textwidth}
\end{frame}
\section{Research Roadmap}
\begin{frame}{Aggregate Computing Framework: Goals and Means}
  \begin{columns}
    \begin{column}[t]{0.6\textwidth}
      \begin{card}[Goals]
        \begin{itemize}
          \item[\success{\faInfo}] \highlight{Functionality} \faArrowRight \, achieving some collective behaviour
          \begin{itemize}
            \item environment monitoring \& control
            \item drone rescues
            \item crowd detection
          \end{itemize}
          \item[\success{\faInfo}] \highlight{Non-Functionality} \faArrowRight \, cost related to the functionality.
          \begin{itemize}
            \item \emph{time efficiency} \faArrowRight \, convergence time of collective specifications
            \item \emph{communication efficiency} \faArrowRight \, amount of data exchanged
            \item \emph{execution efficiency} \faArrowRight \, rounds performed to achieve a collective behaviour
            \item \emph{energy efficiency} \faArrowRight \, communication \faPlus \, execution efficiency
            \item \emph{dependability}
          \end{itemize}
        \end{itemize}
      \end{card}
    \end{column}

    \begin{column}[t]{0.4\textwidth}
      \begin{card}[Means]
        \begin{itemize}  
          \item[\success{\faInfo}] \highlight{Algorithms}
          \item[\success{\faInfo}] \highlight{Execution strategy}
          \item[\success{\faInfo}] \highlight{System structure}
        \end{itemize}
      \end{card}
    \end{column}
  \end{columns}
\end{frame}

\begin{frame}{Aggregate Computing \faPlus \,  Machine Learning}
  \begin{card}[\textbf{Vision}]
    \begin{itemize}
      \item State-of-the-art \faArrowRight \, means devised through \emph{hand-crafted} solution (heuristics, models, \dots)
      \item Machine Learning \faArrowRight \, \emph{augment} (or replace) traditional solution 
      \item[\success{\faThumbsUp}] \highlight{Potential benefit / Opportunities}
      \begin{itemize}
        \item Automatic design of CI
        \item Coherent and high-level specification (through AC programming stack) \faPlus \, learn by doing or from data \faArrowRight \, hybrid programming approach
        \item Improving adaptivity of collective specifications
      \end{itemize}
      \item[\failure{\faThumbsDown}] Challenges
      \begin{itemize}
        \item Distribution \& decentralisation 
        \item Partial observability
        \item Many agent coordinations
        \item Eventual nature of collective computation
      \end{itemize}
    \end{itemize}
  \end{card}
\end{frame}
\begin{frame}{Aggregate Computing \faPlus \,  Machine Learning: Roadmap}
  \centering
  \cardImg{img/roadmap}{0.8\textwidth}
\end{frame}
\begin{frame}[allowframebreaks]{Learning Aggregate Computing algorithms}
  \begin{cardTiny}
    \begin{itemize}
      \item \highlight{Aggregate Computing Algorithms}: Function from Field to Field (Collective Data Structure)
      \item Building blocks: Set of self-organisation patterns for collective application ~\cite{beal2015aggregate-programming}
      \begin{itemize}
        \item G: \emph{gradient-cast} \faArrowRight \,  information sharing outwards a source
        \item C: \emph{collect} \faArrowRight \, information collection inwards a source
        \item S: \emph{sparse choice} \faArrowRight \, distributed leader election
        \item \dots
      \end{itemize}
      \item Each block could have several implementations \emph{reasons} \faArrowRight \, different non-functional
      outcomes:
      \begin{itemize}
        \item time constraints
        \item highly dynamic environment
        \item \dots
      \end{itemize}
      \item[\failure{\faThumbsDown}] hand-crafted building block concerns:
      \begin{itemize}
        \item complex parameter tuning
        \item complex definition in unknown environment
        \item hard to use in non-stationary environment (when should I change the block implementations?)
      \end{itemize}
    \end{itemize}
  \end{cardTiny}
  \begin{card}[The gradient algorithm~\cite{audrito2017ult}]
    \begin{itemize}
      \item A function mapping a Boolean field of sources to the field of minimum distances from those sources
      \begin{itemize}
        \item \emph{progressive}: they take time  to converge to the ``correct'' value
        \item \emph{self-healing}: they can adjust their output following changes in
        their inputs and the system topology
      \end{itemize}
      \item \highlight{Non-functionality} concerns:
      \begin{itemize}
        \item convergence time \faArrowRight \, CRF, BIS, SVD~\cite{audrito2017ult}
        \item output smoothness \faArrowRight \, Flex~\cite{audrito2017ult}
      \end{itemize}
    \end{itemize}
  \end{card}
  \begin{card}[Proposed approach: Program sketching~\cite{solar2008program} / synthesis]
    \begin{itemize}
      \item Designers continue to use high-level block \& API
      \item Part of the programs contains \emph{holes} related to non-functional aspects
      \item A Machine Learning algorithm fills these \emph{holes}
      \begin{itemize}
        \item following a non-functional specification (energy, time, message exchanged, \dots)
        \item using raw experience (i.e., online/continual learning) or through simulations \& data \faArrowRight \, adaptivity
      \end{itemize}
    \end{itemize}
  \end{card}
  \framebreak
  \begin{card}[Preliminary work: fill holes through experience with Q-Learning~\cite{aguzzi2022towards}]
    \begin{itemize}
      \item Aggregate program evaluation combined with RL loop
      \item During the evaluation, the missed part if refined with Q-Learning
      \item \highlight{Schema}: Centralised Traning / Decentralised Execution
      \item \highlight{Future works}: policy generalization issues, deals with continuous state/action, \dots
    \end{itemize}
  \end{card}
  \centering
  \cardImg{img/aggregate-agent-control-architecture-rl.pdf}{0.4\textwidth}
\end{frame}

\begin{frame}[allowframebreaks]{Learning execution strategies and adaptations}
  \begin{cardTiny}
    \begin{itemize}
    \item AC program \faArrowRight \, possibly multiple \emph{execution strategies} that affect
    \begin{itemize}
      \item scheduling of computations
      \item scheduling of communications
      \item retention of messages from neighbours
    \end{itemize}
    \end{itemize}
  \end{cardTiny}
  \begin{card}[Execution strategy]
    \begin{itemize}
      \item Static
      \begin{itemize}
        \item time-based
        \item reactive \faArrowRight \, depends on environment events (messages, sensors, \dots)
      \end{itemize}
      \item Dynamic \faArrowRight \,  adapts the execution choices at
      runtime depending
      \item \emph{Programmable distributed schedulers}~\cite{danilo2021lmcs} \faArrowRight \, a hand-crafted solution to manage collective program scheduling specification
      \item \textbf{Vision} \faArrowRight \, dynamic scheduling via learning or evolutionary approach following \emph{high-level} system goals \faArrowRight\, towards \emph{green autonomic computing}
    \end{itemize}
  \end{card}
  \begin{card}[Preliminary work: Addressing Collective Computations Efficiency through RL]
    \begin{itemize}
      \item RL agent observes the output of a collective computation \emph{locally}
      \item Reward function: try to maximise the trend between \emph{converge-time} and \emph{consumption}
      \item Action: next-wake-up time
      \item \highlight{Results}: agents eventually find policies that reduce \emph{the collective} power consumption
      \item \emph{Future works}: more robust evaluation, online learning, \dots
    \end{itemize}
    \cardImg{img/image-rl-500-plain.pdf}{\textwidth}
  \end{card}
\end{frame}

\begin{frame}[allowframebreaks]{Learning system structures and re-structuring}
  \begin{columns}
    \begin{column}[c]{0.65\textwidth}
      \begin{cardTiny}
      \begin{itemize}
        \item Aggregate systems \faArrowRight \, logical network  of logical devices that perform rounds
        \item Extremely flexible model \faArrowRight \, pulverised architecture~\cite{casadei2020pulverization}
        \item The same program could be executed in several deployment
        \begin{itemize}
          \item Collective specification does not depend on concrete IT Networks
          \item Enable opportunistic code movement (from edge to cloud and vice versa)
        \end{itemize}
        \item[\failure{\faThumbsDown}] Complex deployment definition for a given goal (simulations)
        \item[\success{\faThumbsUp}] \textbf{Vision}: inject ML to let the system learns the effective deployment policy \faArrowRight \,  improving the efficacy and
        efficiency of aggregate applications.
      \end{itemize}
    \end{cardTiny}
  \end{column}
  \begin{column}[c]{0.3\textwidth}
    \cardImg{img/deployments.jpg}{\textwidth}
  \end{column}
  \end{columns}
\end{frame}


\begin{frame}{Conclusion}
  \begin{card}[Recap]
    \begin{itemize}
      \item Aggregate computing \faArrowRight \, effective approach to engineer Collective Intelligence
      \item \highlight{Vision} \faArrowRight \, improve aggregate application (functional \& non-functional aspects) through learning:
      \begin{itemize}
        \item Algorithms
        \item Execution strategies
        \item deployments
        \item[\success{\faThumbsUp}] Preliminary work shows good results
      \end{itemize}
      \item Opportunities:
      \begin{itemize}
        \item Automatic desing of CI~\cite{szuba2001formal}
        \item Green autonomic computing
      \end{itemize}
      \item Challenges
      \begin{itemize}
        \item Many-agents settings
        \item Distribution
        \item Partial observability
        \item \dots
      \end{itemize}
    \end{itemize}
    
  \end{card}
\end{frame}
\begin{frame}[allowframebreaks]
  \frametitle{References}
  \printbibliography
\end{frame}
\end{document}
